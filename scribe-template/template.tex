\documentclass[usletter]{article}
\usepackage{graphicx}
\usepackage{amsfonts}
\usepackage{amsthm}
\usepackage{amsmath}
\usepackage{amssymb}
\usepackage{hyperref}
\hypersetup{
    colorlinks=true,
    linkcolor=blue,
    filecolor=magenta,      
    urlcolor=cyan,
}
\usepackage{scribe}
\usepackage[margin=1.5in]{geometry}

\begin{document}


\makeheader{Sanket Kanjalkar, Yunqi Li}                              % your name
           {October 09, 2019}                          % lecture date
           {11}                                       % lecture number
           {Fully Homomorphic Encryption I}  % lecture title

\newcommand{\floor}[1]{\left\lfloor #1 \right\rfloor}
\newcommand{\ceil}[1]{\left\lceil #1 \right\rceil}

\section{Construction of Fully Homomorphic Encryption:}
As described previousy, FHE is an ecryption scheme(Gen, Enc, Dec) 
with an additional algorithm called Eval. In particular, we want to construct 
such a Eval namely for two operations, Addition and Multiplication. Constructing 
such a FHE scheme which must work for \textbf{all} functions $f$ might seem like daunting task, 
but can use the following fact to ease our task.
\begin{fact}
It turns out that all functions can be expressed by arithematic circuits consisting
of only addition and multiplication gates. Therefore, we only to implement our FHE operations for 
addition and Multiplication. We can recursively compute every gate  in the Arthematic circuit 
homomorphically to get the output of the function. 
\end{fact}

\begin{definition}
An arithmetic circuit over a field $Z_q$ is a directed
acyclic graph whose vertices are called gates. Gates of incoming degree 0 are inputs to
the circuit. All other gates are labelled $+$ or $x$. 
\end{definition}
We usually consider Arithematic circuits of fan-in 2 circuits, in which 
case all of the + and × gates have in-degree 2.

\begin{remark}
Even though Fully Homomorphic encryption scheme is our actual goal, in practice we also consider 
a simplification leveled fully homomorphic encryption scheme. Leveled FHE does not allow us to 
compute aribritary functions $f$ but only functions with apriory known depth $d$. Informally, when 
we already know what is the most complex(in terms of depth of the arithematic circuit) and use that 
in the construction of our FHE
\end{remark}

Let us try to the simplest possible way to build a FHE for the addition operation. 
For simplicity, let us consider that we want to single bit numbers 
and output a single bit number. This is equivalent to implementing the XOR operation. 

\section{FHE: Addition operation}

	Consider a encryption of message $\mu_1$ under the public key ($s^{T}A + e^{T}$, $A$). We call 
$s^{T}A + e^{T}$ as $b$. One intuitive naive way to implementing addition might be addition of 
ciphertexts

$$c_1 = ( Ar_1, br_1 + \mu_1\floor{\frac{q}{2}} )$$
$$c_1 = ( Ar_2, br_2 + \mu_2\floor{\frac{q}{2}} )$$
$$c_{add} = c_1 + c_2 = ( A(r_1 + r_2), b(r_1 +r_2) + (\mu_1 + \mu_2)\floor{\frac{q}{2}} )$$

It is possible to extend this to multi-bit XOR outputs by simply repeating 
the circuit multiple times. However, it would only help in computing XOR for two $k$ bit numbers. 
Let us try to decrypt the ciphertext $c_{add}$ and check what it decrypts to:

% Copy the decryption equation here.


So applying the decryption algorithm we get $\mu_1$ + $\mu_2$ given the total error
is small ($|e_1 +e_2| \leq q/4$). The important observation to note here is to perform addition on 
two ciphertexts we need to assume hardness of LWE for stronger security parameters. 
Therefore, if we want to perform $l$ addition operations, we would have to keep our $max(e_i) \leq \floor{\frac{q}{2}}/l$.   

\begin{corollary}
To compute addition of $k$ bit numbers $\mu_1$ and $\mu_2$ we must change our 
encryption scheme to the where the factor which is multiplied to the plaintext should be $\frac{q}{2^(k+1)}$. 

$$c_1 = ( Ar_1, br_1 + \mu_1\floor{\frac{q}{2^{k+1}}} )$$
$$c_1 = ( Ar_2, br_2 + \mu_2\floor{\frac{q}{2^{k+1}}} )$$
$$c_{add} = c_1 + c_2 = ( A(r_1 + r_2), b(r_1 +r_2) + (\mu_1 + \mu_2)\floor{\frac{q}{2^{k+1}}} )$$

On the basis of the similar argument described above, the error of the equations for 1 addition be 
constrained by $e_i \leq \frac{q}{2^{k+1}}$.
\end{corollary}
\begin{remark}
It is also possible to implement a similar addition for the private key ecryption part scheme 
using LWE. That is, adding two ciphertexts $c_1$ and $c_2$ corresponding to $\mu_1$ and $\mu_2$
would also result in encryption of message $\mu_1 + \mu_2$.
\end{remark}

\begin{remark}
A natural question which arises from the above discussion is about the similarility XOR operation 
and addition operations.   
\end{remark}

\section{Towards FHE multiplication}


\begin{figure}
\begin{center}
\includegraphics[width=0.4\textwidth]{triangle-circle}
\end{center}
\caption{A triangle and a circle.}
\label{fig:triangle-circle}
\end{figure}



\section{Mathematical environments}

For your convenience, the scribe note style file comes
with the following mathematical environments
predefined: theorem, lemma, corollary, proposition,
fact, claim, definition, example, assumption, remark,
conjecture, open problem, problem. The environments are
illustrated below.  Please limit yourself to these
environments.

\begin{theorem}
Statement here 
\end{theorem}

\begin{lemma}
Statement here
\end{lemma}

\begin{corollary}
Statement here
\end{corollary}

\begin{proposition}
Statement here
\end{proposition}

\begin{fact}
Statement here
\end{fact}

\begin{claim}
Statement here
\end{claim}

\begin{definition}
Statement here
\end{definition}

\begin{example}
Statement here
\end{example}

\begin{assumption}
Statement here
\end{assumption}

\begin{remark}
Statement here
\end{remark}

\begin{conjecture}
Statement here
\end{conjecture}

\begin{openproblem}
Statement here
\end{openproblem}

\begin{problem}
Statement here
\end{problem}


\noindent
Note that \LaTeX\ automatically numbers these
environments within the lecture number (\thelecture\ in
this case).  The same applies to the numbering of pages
(this page being page \thepage), figures
(Figure~\ref{fig:triangle-circle} above), and
equations:
\begin{align}
a = a_1+a_2+\cdots+a_n.
\end{align}
\noindent
For proofs, use the provided {\tt proof} environment,
illustrated below.

\begin{proof}
Proof goes here.
\end{proof}

\section*{Acknowledgement}
These scribe notes were prepared by editing a light modification of the template designed by Alexander Sherstov.
%retain this acknowledgement in all scribe notes.

\bibliographystyle{abbrv}
\bibliography{template}

\end{document}
